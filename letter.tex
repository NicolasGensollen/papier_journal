
\documentclass{letter}
\makeatletter
\newenvironment{thebibliography}[1]
     {\list{\@biblabel{\@arabic\c@enumiv}}%
           {\settowidth\labelwidth{\@biblabel{#1}}%
            \leftmargin\labelwidth
            \advance\leftmargin\labelsep
            \usecounter{enumiv}%
            \let\p@enumiv\@empty
            \renewcommand\theenumiv{\@arabic\c@enumiv}}%
      \sloppy
      \clubpenalty4000
      \@clubpenalty \clubpenalty
      \widowpenalty4000%
      \sfcode`\.\@m}
     {\def\@noitemerr
       {\@latex@warning{Empty `thebibliography' environment}}%
      \endlist}
\usepackage[]{algorithm2e}
\newcommand\newblock{\hskip .11em\@plus.33em\@minus.07em}
\makeatother
\signature{Vincent Gauthier}
\address{Vincent Gauthier \\ Institut Mines Telecom \\ Nano Innov, Saclay \\ France}
\begin{document}

\begin{letter}{Dr. Jianhui Wang \\ Editor-in-Chief \\ Journal Transactions on Smart Grid } % Addressee name and address


\opening{ Dear Dr. Jianhui Wang,}


The authors would like to thank you for giving us the opportunity to revise and resubmit this manuscript. We resubmit this revision before the agreed upon deadline, December 21, 2015. We appreciate the time and details provided by each reviewer and by you and have incorporated the suggested changes into the manuscript to the best of our ability. The manuscript has certainly benefited from these insightful revision suggestions.


See bellow our replies to the reviewers’ comments:


\textbf{Reviewer: 1}


\textbf{Comment 1:} However I can see a case study is presented and the results are summarized in the last section, but I recommend you to give more detail about the case study conditions. I can see so many details are presented in the main body of the manuscript but it is difficult to track till end of the paper.


\textbf{Comment 2:} I do recommend you also to have a general schematic of the system to make it clear the elements of your system.


\textbf{[Answer to 1 and 2]} We agree with the reviewer that a general schematic would help the reader. In order to illustrate the overall process we added a figure (Figure 1) in the article. This figure explains the main steps from the simulation of the agents’ production time series to the formation of the coalitions. We also added a table of the main notations that might help the reader.


\textbf{Comment 3:} In fig. 5, how does the sampling frequency affect the volatility? How does the sampling frequency of your data affect the variance? There are some paper that demonstrates the effectiveness of the available resources data resolution on the expected variability of the renewable energy systems, I think this could be interesting if you can provide some information in this regard.


\textbf{[Answer]} The results presented in this paper come from hourly sampled weather data. Therefore, all the correlations between the agents are derived from hourly sampled time series. This low frequency sampling has the effect of hiding short time intermittencies that are present in both wind speed and solar irradiance real dynamics. According to \cite{Anvari2015}, “extreme events in the solar irradiance and wind power fluctuations are much more frequent in short time scales” than expected with a Gaussian process. This is especially true and understandable for solar irradiance that can change drastically in case of clouds passing in a sunny sky for instance. Although we did not, in this paper, study how the correlation structure evolves with the sampling frequency of the data, we believe that the variance of the power distribution of the coalitions will increase with the sampling frequency. Nevertheless, as our algorithm tries to minimize intra-coalition correlations, we believe that short time fluctuations will tend to impact less “our” coalitions than the ones obtained with the other presented strategies. 


\textbf{Comment 4:} What is the affect of using stochastic modeling of the resources on the coalitions resulting from the proposed method? Using a real data set is a good case study, but in the conclusion you mentioned "the coalitions resulting from our algorithm better withstand losses of agents", did you do any comparison? It is not clear how you came up to this conclusion. I think you should have an apple-to-apple comparison, stochastic modeling can help you.


\textbf{Comment 5:} There are several parameters that makes the systems fail, in page 7 what do you mean by node failure? Do you mean lack of resources OR failure of power electronics interfaces of renewable energy systems? Please clarify that in the paper.


\textbf{[Answer to 4 and 5]} Lack of resources is a key issue of this paper. However in figure 8, we specifically concentrate on the power electronics or line failures using a simplified algorithm. More precisely, we want to study how coalitions resulting from different formation algorithms respond to failures. Failures are considered random and when some agent fails, it is deleted from its coalition. This has consequences on the production of the coalition that may cause the violation of one of the market rules (stability and minimum production), which in turn will lead to the deletion of the coalition from the market. We used this method in order to compare how the coalition structures obtained with the three different algorithms (“correlated”, “de-correlated”, and “random”) respond to this type of failure. We concluded that the coalitions resulting from our “de-correlation” algorithm were able to better sustain agents loses than the two others. We agree with the reviewer and added, for clarity, a section in the appendix (see section Resilience algorithm) that gives the pseudo code of the process discussed above (reproduced here as algorithm \ref{alg:algo4}).

\begin{algorithm}
	\KwData{$ CS = \left\{ S_1,S_2,...,S_k \right\} $ Coalition structure \;}
	
	$ pool = \cup_{k \in N_{COAL}} S_k $ \; 
	\While{ $ pool \neq \emptyset $}{
		Select randomly agent i from pool\;
		i fails $ \Rightarrow $ Remove i from its coalition : $ S_k \leftarrow S_k - \{i\} $ \;
		$ pool = pool - \{i\} $\;
		\If{ $ Pr \left[ P_{S_k} < P^{MIN} \right] > \phi $}{
			$S_k$ fails $ \Rightarrow $ Remove $S_k$ from the market
			} 
		}
	\caption{Random failures algorithm}
\label{alg:algo4}
\end{algorithm}

 We also added the following lines in the article: 

\textit{“We consider here the case of random failures of the power electronics of some agents that has the consequence of preventing them from participating in the market. Therefore, the notion of resilience we will use in the following can be seen as the ability of the coalition structures to inject stable power in the grid when some of its internal agents are removed.”}


\textbf{Comment 6:} Fig. 5 is a great visualization of the problem, but how are you going to come-up to a final decision? Can you provide formulation to make decision using decision theory? I think this will help and support your great work.


\textbf{[Answer]} We thank the reviewer for this comment. In figure 5 we represented the distribution of the coalition structures using Monte Carlo sampling as comparison between the results of the three algorithms. All algorithms need a desired number of coalitions that they should output. In our work, the decision on the number of coalitions is therefore assumed to come from the user, and is not investigated. Given the desired number of coalitions, the algorithms use the utility function (eq. 11) as the decision criterion between the different coalition structures (all containing the same number of coalitions). This can be seen in more details in the pseudo-code of the algorithms in the appendix (see section Algorithms of the appendix). We also reproduce the pseudo code of the greedy optimization below (alg. \ref{alg:algo1}). Our algorithm for instance starts with seeds in the correlation graph and expands them as long as the utility is improved. It stops when no agent is able to increase the coalitions’ utilities.


\begin{algorithm}
 \KwData{$P_{i}$ series,\\ Grid policy $ (P^{MIN},\phi) $,\\ Desired number of coalitions $ N_{COAL} $,\\ size of starting cliques k}
 \KwResult{ $ CS = \{ S_{1},...,S_{N_{COAL}}\} $ }
 Compute $ G_{2}^{\epsilon^{\star}} $ \;
 Find the $ N_{COAL} $ cliques in $ G_{2}^{\epsilon^{\star}} $\;
 \While{$ \mathcal{U}(CS) $ is improving}{
 	\For{each clique}{
 		Find $ i^{\star} $ \;
 		\If{ $ \delta_{clique}(i^{\star}) \geq 0 $ }{
 			$ clique \leftarrow clique \cup \{i^{\star} \} $ \;
 			}
 		\If{ $ \exists j \in clique,\ s.t\ \delta_{clique}(j) < 0 $}{
 			$ clique \leftarrow clique - \{j \} $ \;
 			}
   		}
  	}
 \caption{Local greedy optimization algorithm}
 \label{alg:algo1}
\end{algorithm}

\textbf{Comment 7:} In general the paper is in well format, however it needs some grammar revision as well as proof reading, I saw couple of errors such as the question marks in page 3, column 2.


\textbf{[Answer]} We tried to correct all the remaining errors in the paper.



\textbf{Reviewer: 2}

\textbf{Comment 1:} Using a Table with ALL notations at the very beginning of the manuscript.

\textbf{[Answer]} We respond to the reviewer’s comment and added a table of the main notations in the article (figure 2). Because of space limitation, we were not able to put all the notations in this figure. Therefore, we also added a table of all the notations in the appendix.


\textbf{Comment 2:} Making clear the fact which grid/system operator is referred to (distribution vs transmission)

\textbf{[Answer]} In this paper, we consider aggregations of prosumers located in the distribution network as stated in the introduction:
“Since production could be located down to the very end of the distribution networks, nodes that were simply pure loads yesterday could behave tomorrow as generators or loads, further complicating energy usages. In this paper, we focus on these "nodes" in the distribution network that can produce and consume electricity.”

For clarity, we added the following line in the beginning of section 4:

\textit{“We consider a set A of N prosumers of the distribution network.”}


\textbf{Reviewer: 3}

\textbf{Comment 1:} In Section III, is load uncertainty included in the generation of prosumer patterns? If yes, what load patterns were used? What happens if the load is so high that prosumers cannot generate energy?


\textbf{[Answer]} In this paper, we generate consumption historical series using real temperature traces (see section Consumption of the appendix). More precisely, we decompose the consumption of a prosumer into a heating and an electronic consumption term.

\begin{equation}
P_{i}^{D}(t) = \mathcal{F}_{i}^{heat}(\tau(t), t) + \mathcal{F}_{i}^{elec}(t)
\end{equation}
where $ \mathcal{F}_{i}^{heat}(\tau(t), t) $ is the power curve that maps the temperature $ \tau(t) $ to a heating consumption, and $ \mathcal{F}_{i}^{elec}(t) $ computes the consumption of agent i (other than heating) at a given hour of the day. The purpose of this decomposition is to reproduce both seasonal patterns as well as daily patterns of consumption in a simple model.

In the simulation, all agents have a desired inside temperature $ T_{i} $, supposed to be a constant for simplification. By using thermodynamic laws $ \mathcal{F}_{i}^{heat}(\tau(t), t) $ can be approximated by :

\begin{equation}
\mathcal{F}_{i}^{heat}(\tau(t), t) = \frac{B_{i}}{R_{i}} \left[ T_{i} - \tau(t) \right]
\end{equation}
where $ B_{i} $ is the surface of thermal exchanges for agent i and $ R_{i} $ is their thermal resistance. 

We further denote by $ \Omega_{i} $ the maximum consumption possible for agent i, which is basically the sum of all its appliances powers. We also denote by $ \omega_{i}(t) = \{ \omega_{i}(t_{0}),...,\omega_{i}(t_{24}) \} $ the vector of the average fraction of $ \Omega_{i} $ used for each hour ($\forall i,\ \forall t,\ \omega_i(t) \in [0,1] $). We can therefore write :

\begin{equation}
\mathcal{F}_{i}^{elec}(t) = \Omega_{i} ( \omega_{i}(t) + \epsilon )
\end{equation}
where $ \epsilon $ is a noise term drawn from a normal distribution. The vector $ \omega_{i}(t) $ enables us to easily differentiate agent consumption behaviors. Business or residential areas for instance can be easily distinguished with this kind of model.

At time t, if the consumption of a prosumer is larger than its production, then this prosumer will consume energy from its coalition rather than contributes to the production. If all prosumers of a same coalition are strongly correlated and tend to behave in the same way this results in a very unstable output production. However, in the case of uncorrelated prosumers, over-production of some of them might tend to compensate the under-production of the others, yielding a more stable output. In the case of a prosumer that almost never produce, its marginal utility would be very small or negative, such that it will not be selected in the expansion phase.


\textbf{Comment 2:} The selection of the parameter alpha of the coalition utility function [eq. (14)] relies heavily on the Gaussianity assumption for the generated net power, which may not hold.


\textbf{[Answer]} We agree with the reviewer’s comment. The purpose of the alpha parameter is to control to some extent the expected sizes of the coalitions. In order to obtain an analytical formulation for this parameter, we used the assumption that the distributions of the coalitions’ production were Gaussians. When the number of agents in the coalitions increases, the distribution of the coalition will tend to a normal distribution for independent agents. Although independence is generally not true and is not required for our method, we found that using this assumption for tuning alpha gives good results in practice. Other parameter selection methods such as grid search could also be used although not presented in the article.


\textbf{Comment 3:} The selection of the threshold epsilon for the decorrelation graph [eq. (15)] appears to incur exponential complexity, as it requires availability of cliques of the graph.


\textbf{[Answer]} We agree with the reviewer’s point that finding cliques in graphs is usually a hard problem. In order to overcome this difficulty when the number of nodes is not small, we restrict the algorithm to triangles rather than cliques. Finding $ \epsilon{\star} $ through (eq.(15)) with $ k = 3 $ (looking only at triangles) requires to count the number of triangles in a graph. There exist several techniques to achieve this (see \cite{Schank2001}). One of the most straightforward is:

\begin{equation}
 N_{triangles} = \frac{1}{6} Tr \left[ A^3 \right] 
\end{equation}
 
Where A is the graph adjacency matrix. This leads to a counting algorithm which runs in $ O(n^q) $ where q is the matrix multiplication exponent (considered to be less than 2.4).

For finding the starting seeds for the clique expansion, we need not only to count the number of triangles, but we also have to enumerate them. Again, several techniques exist such as node-iterator or edge-iterator (more information in \cite{Schank2001}). 

We clarify this point by adding the following text in section IV-A:

\textit{“In practice, since finding cliques requires exponential time, we use triangles [14] (k = 3 in eq. 15) rather than cliques as soon as the number of nodes is not small.” }


\textbf{Comment 4:} Implementation of the algorithm also requires the correlation coefficient between any two prosumers, which may as well not be available.


\textbf{[Answer]} We agree with the reviewer’s comment. In order for our algorithm to work, we need the correlation coefficients between any two prosumers that can be obtained from historical time series. Having these series is an assumption of our work.


\textbf{Reviewer: 4}

\textbf{Comment 1:} The reviewer is wondering if the research question is application for the general context or only for specific environment. In general, it would be expected that the prosumers have their own freedom to choose their aggregator, similar way to have the electricity contract with the energy supplier. Forcing the end users to be in dedicated coalition would be realistic.


\textbf{[Answer]} In this paper, we indeed do not consider agent freedom to choose what coalition to join. We only consider the maximization of a global utility function that seeks de-correlation inside the coalitions. Perhaps, a more realistic framework based on Game theory will consider a utility function per agent with different parameters. Each (rational) agent would then focus on the maximization of its own profits, and coalition structures might emerge from series of split and merge actions. Although not considered in the present work, we believe that building a game which would take into account the stability of power injections through the correlation structure might be possible.


\textbf{Comment 2:} The reviewer was not convincing about the statement in page 4, line 21-24, “If S always procedures… a higher contract value”. The reviewer believes that if S deviates from the scheduled value, even if higher extra-production it should still have the penalty cost.


\textbf{[Answer]} We agree with the reviewer that over-producing is also problematic toward the grid’s stability. And coalitions should pay penalties if deviations (positive or negative) from the contract occur. However, in our framework, we make the distinction between the available power of a coalition, and the power it really injects in the grid. Although the latter has to be equal to the contract, we allow the former to be larger since the difference can be used for storage or other usages. We changed the misleading statement to:

\textit{“If S has an available production always greater than $ P_{S}^{CRCT} $, it is losing some gains since it could have announced a higher contract value.”}


\textbf{Comment 3:} The reviewer is doubting about the interaction model developed in the paper, specifically mentioned in the second column of page 4. Why should the grid operator concern about the coalition of prosumers with its probability of under-producing? Normally saying, under-producing should not cause any grid issue while significant over-producing from the end users would might cause current congestions or voltage violation for the distribution network.


\textbf{[Answer]} In this paper, we consider the case of a large penetration of renewables in the distribution network. A significant portion of the production is assumed to come from these coalitions of prosumers, in such a way that the grid operator is scheduling the plants according to the contract values. If coalitions do not satisfy their contracts by under-producing, it might be the case that the grid operator has to re-balance consumption and production by turning on fast ramping generators. This might generates high operational costs for the grid operator.


\textbf{Comment 4:} Keywords are missing in this paper!


\textbf{[Answer]} We have considered the reviewers request and added keywords at the beginning of the article.


\textbf{Comment 5:} The introduction part is too long with well-known context that can be found in many previous works. It is recommended to refer more to the literature and shorten the introduction part to go direct to the research problem/questions


\textbf{[Answer]} We agree with the reviewer’s comment and we shortened the introduction.


\textbf{Comment 6:} Terminologies of reliability and stability have been used similarly here. Actually, power system researcher would consider stability with different meaning.


\textbf{[Answer]} In this paper, we used the word “stability” to characterize the “smootheness” of the power output of the coalitions. In this paper, this “smootheness” appears as a consequence of the variance of the coalitions’ power distributions. Therefore the “stability” and “reliability” concepts of our paper are strongly connected but we wanted to make the distinction. Naturally, we agree with the reviewer that these terms are used here in a special way, and might be considered differently by power system researchers.

\bibliographystyle{unsrt}
\bibliography{letter}

\end{letter}
\end{document}